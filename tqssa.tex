\documentclass[12pt]{report}

\usepackage{hyperref}

\newcommand{\link}[2]{\href{#1}{#2}}

% \usepackage[utf8]{inputenc}
% \usepackage{mathtools}
\usepackage[parfill]{parskip} % line skip paragraphs
\linespread{1.5} % linespacing 1.5

\usepackage{amsmath} % good standards

\begin{document}

\bibliographystyle{plain}

\title{TQSSA}
\author{Oskar Thor\'{e}n}

\maketitle

\chapter{TQSSA}

The enzyme-substrate-complex schema

\begin{equation}
E + S \leftrightarrow C \rightarrow E + P
\end{equation}

where $k_1$, $k_{-1}$ and $k_2$ are parameters belonging to the respective arrows, can be described by the follow set of differential equations.

\textbf{TODO:} Show derivation.

\textbf{TODO:} Fill out rest of the ODEs.

\begin{equation}
\frac{dS}{dT} = ... \\
\frac{dC}{dT} = ...
\end{equation}

with initial conditions $(S,E,C,P) = (S_0, E_0, 0, 0)$

\textbf{TODO:} Fill out description of QSSA.
The Quasi Steady State Assumptions tells us that we can approximate ... assuming ...

TQSSA is an extension of the standard QSSA by the use of a change of variable.

When we have too much enzyme for the standard QSSA to be valid, we can sometimes replace the free substrate concentration by total substrate concentration, as follows.

\begin{equation}
\overline{S} = S+C
\end{equation}

\textbf{TODO:} Reference and derivation of ODE.
This changes the previous ODEs in ... to the following system of differential equations.

\begin{align*}
\frac{d\overline{S}}{dt} &= -k_2C, \\
\frac{dC}{dt} &= k_1[(E_0 - C)(\overline{S}-C)-K_m C],
\end{align*}

with the same initial conditions as before.



This is the criteria for its applicability
\begin{equation}
\frac{k_2 E_0}{k_1(E_0+S_0+K_m)^2} \ll 1
\end{equation}

\bibliography{references}

\end{document}
