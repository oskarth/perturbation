\documentclass[12pt]{report}

\usepackage{hyperref}

\newcommand{\link}[2]{\href{#1}{#2}}

% \usepackage[utf8]{inputenc}
% \usepackage{mathtools}
\usepackage[parfill]{parskip} % line skip paragraphs
\linespread{1.5} % linespacing 1.5

\begin{document}

\bibliographystyle{plain}

\title{Singular Pertubation Theory and TQSSA}
\author{Oskar Thor\'{e}n}

\maketitle

\chapter{Singular perturbation theory}

\section{History and introduction}

\textit{It causeth my head to ache} -- Newton on the orbits of the moon. Three
body problem. Discovery Neptunus, but also Vulcan. Poincare and Prandtl.

Many phenomena can be modelled by ODEs, but only a subset of ODEs can be solved
exactly. We need approximate solutions that are justified in a rigorous manner.

\section{Simple regular perturbation}

\begin{equation}
  z^2 - 2z + \epsilon
\end{equation}

Assume we don't know exact solution to this quadratic equation. For small
$\epsilon$, how does the solution change? If we set $\epsilon$ to zero we get
two solutions.

In this case we know that $z=1 \pm \sqrt{1 - \epsilon}$, so it's easy to see what happens to the solution for any given small $\epsilon$, solutions will be around 0 and 2.

Expand $\sqrt{1 - \epsilon}$ as a Taylor series: $1 - \frac 1 2 \epsilon - \frac 1 8 \epsilon^2$

This gives $z_1 = \frac 1 2 \epsilon + O(\epsilon^2)$ and $z_2 = 2 - \frac 1 2 \epsilon + O(\epsilon^2)$ etc.

\section{Simple singular perturbation}

\begin{equation}
  \epsilon z^2 - 2z + 1
\end{equation}

In this example we have $z = \frac{(1 \pm \sqrt{1-\epsilon})}{\epsilon}$. Note
that $\epsilon=0$ removes one of the roots. This is indicative of a singular
perturbation problem - we are missing something when we remove the error term.

Unlike in the last section, $z(\epsilon) = \frac 2 \epsilon + ...$, obviously a
problem here.

Perturbation series $z(\epsilon) = a_0 + a_1 \epsilon + a_2 \epsilon^2 + ... +
R^{n+1}, R^{n+1} = O(\epsilon{N+1})$

Procedure for this:

1. Substitute
2. Expand
3. Collect
4. Substitute original IC/BC
5. ...

\section{ODE example}

ODE with BC. Can't fulfil both conditions at once. So two solutions, oe inner
$O(\epsilon)$ and one other $O(1)$. Then combine (match) these to get a uniform
solution.

Read BV chapter for this.

\chapter{TQSSA}

\section{Biological problem}

Thing and enzyme. Combines fast. QSSA two time scales. TQSSA similar.

\section{Scaling}

Scale dimensionless. Two periods. Basics of dimensionaless.

\section{Inner solution}

$0 < \epsilon \leq 1, \epsilon = \frac{t_{\epsilon}}{t_{s}}$

\section{Outer solution}

\section{Matching and uniform approximation}

\section{Applying it}

See Verdugo.

\bibliography{references}

\end{document}
