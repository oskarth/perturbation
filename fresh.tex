\documentclass[12pt]{article}
\usepackage[parfill]{parskip}
\usepackage{amsmath}
\usepackage[nottoc,numbib]{tocbibind} % show bib in toc
\linespread{1.5}

\bibliographystyle{plain}

\title{Perturbation theory and TQSSA}
\author{Oskar Thor\'{e}n}
\date{2015-02-15}

\begin{document}

\nocite{*} % include all references
\maketitle

TODO: Write abstract and title.

\begin{abstract}
Lorem ipsum dolor sit amet, consectetur adipisicing elit, sed do
eiusmod tempor incididunt ut labore et dolore magna aliqua. Ut enim ad
minim veniam, quis nostrud exercitation ullamco laboris nisi ut
aliquip ex ea commodo consequat. Duis aute irure dolor in
reprehenderit in voluptate velit esse cillum dolore eu fugiat nulla
pariatur. Excepteur sint occaecat cupidatat non proident, sunt in
culpa qui officia deserunt mollit anim id est laborum.
\end{abstract}

\clearpage
\tableofcontents
\clearpage

\section{Introduction}

Perturbation theory has its roots in celestial mechanics and aerodynamics. There
are two particularly interesting victories the field has seen. The first was the
discovery of Neptune, and the second is the theoretical foundation for
aerodynamics. We will give a short account of these in the history section.

The promise of perturbation theory is that it allows us to solve a
larger class of problems than we could otherwise with analytical methods. It
does this by giving us approximate, but rigorous, solutions. In this article we
will guide the reader from a very simple and familiar example, a quadratic
equation, all the way to a useful research example in system biology: the total
quasi steady state in enzyme kinetics.

The article has been written with the goal that any student with a basic
understanding of calculus, differential equations and linear algebra will be
able to follow along. All the relevant biology and perturbation theory will be
explained as we go along.

\subsection{History}

TODO: Write a subsection on the history of perturbation theory. From
Scholarpedia.

\section{A regular perturbation}

We are going to start with probably the simplest non-trivial example imaginable:
a quadratic equation.

\begin{equation}
x^2 - 2x + \epsilon = 0
\end{equation}

We know how to solve this analytically. The roots of the equation are $x_1 = 1 +
\sqrt{1 - \epsilon}$ and $x_2 = 1 - \sqrt{1 - \epsilon}$.

The case we are interested in is the one where $\epsilon$ is very small. In this
case, we can see that setting $\epsilon=0$ produces the roots $x=0$ and $x=2$
respectively. In fact, for any given small $\epsilon$ we notice that solution
changes very little. In that sense it's a very "boring" and predictable problem.

How can we make this observation more rigorous? One way is to rewrite the part
of the solution containing $\epsilon$ as a Taylor series. Recall that a Taylor
series for a function $f(x)$ around a looks like follows.

\begin{equation}
f(x) = \sum_{n=0}^{\infty} \frac{f^{n}(a)}{n!} (x-a)^n
\end{equation}

The Taylor series for $f(\epsilon) = (1 - \epsilon)^{1/2}$ around 0 is thus.

\begin{equation}
\sqrt{1 - \epsilon} = 1 - \frac{\epsilon}{2} - \frac{\epsilon^2}{8} + O(\epsilon^3)
\end{equation}

If we take the limit of this as $\epsilon \to 0$, it's obvious that our
intuition is correct. That is, a small change in $\epsilon$ only brings about a
small change in the solution.

\begin{align}
x_1 &= \frac{\epsilon}{2} + \frac{\epsilon^2}{8} + O(\epsilon^3), \\
x_2 &= 2 - \frac{\epsilon}{2} - \frac{\epsilon^2}{8} + O(\epsilon^3)
\end{align}

What is the point of all this? We said in the beginning that perturbation theory
allows us to solve a larger class of problems, but what we have done so far
hasn't given any indication of that being true. After all, we already have an
analytical formula for the quadratic equation.

Let's start over, but this time let's assume we don't have the quadratic formula
at our disposal. We will outline a method which would allow us to get arbitrarly
good approximations for polynomials of any degree.

Let's assume the solution of (1) can be expressed on the form of a power series
of $\epsilon$.

\begin{equation}
\sum_0^{\infty} a_k \epsilon^k = a_0 + a_1 \epsilon + a_2 \epsilon^2 + ...
\end{equation}

We will now insert this power series into (1), and then expand that expression
in terms of powers of $\epsilon$. We only have to do this for the first few
terms of the power series, as we will see in the end. If we want to we can
always add more terms and get a more accurate approximation.

\begin{equation}
(a_0 + a_1 \epsilon + a_2 \epsilon^2 + ...)^2 - 2(a_0 + a_1 \epsilon + a_2
\epsilon^2 + ...) + \epsilon = 0
\end{equation}

We expand the expression using Big-Oh algebra. For example, $(a_0 + a_1 \epsilon
+ a_2 \epsilon^2)^2 = a_0^2 + 2 a_0 a_1 \epsilon + (a_1^2 + 2 a_0 a_2) \epsilon^2 + O(\epsilon^3)$.

The whole expression becomes the following.

\begin{equation}
a_0^2 - 2 a_0 + (2 a_0 a_1 - 2 a_1 + 1)\epsilon + (a_1^2 + 2 a_0 a_2 - 2 a_2)
\epsilon^2 = O(\epsilon^3), \epsilon \to 0
\end{equation}

Now we turn the problem of determining the coefficients $a_0$, $a_1$ etc. Since
$\epsilon$ is a variable here rather than a parameter, we see that the
coefficients before each power of $\epsilon$ separately all have to be equal to
zero. For example, $a_0^2 - 2 a_0 = 0$ when $\epsilon \to 0$, and if we remove
that part and divide the rest by $\epsilon$ we get the same for the next
coefficient, etc. We thus get the following system of equations for solving the
coefficients.

\begin{align}
a_0^2 - 2 a_0 &=0, \\
2 a_0 a_1 - 2 a_1 + 1 &= 0, \\
a_1^2 + 2 a_0 a_2 - 2 a_2 &= 0
\end{align}

Solving these equations in turn gives us $a_0 = 0$ or $a_0 = 2$. For $a_0 = 0$ we
have $a_1 = \frac{1}{2}$ and $a_2 = \frac{1}{8}$. For $a_0 = 2$ we have $a_1 = -
\frac{1}{2}$ and $a_2 = - \frac{1}{8}$.

We said at the beginning that we assume the solution is on the form of a power
series of $\epsilon$. We have two options for the coefficients, and these
corresponds to the two approximate solutions.

\begin{align}
x_1 &= \frac{1}{2} \epsilon + \frac{1}{8} \epsilon^2 + O(\epsilon^3), \\
x_2 &= 2 - \frac{1}{2} \epsilon - \frac{1}{8} \epsilon^2 + O(\epsilon^3)
\end{align}

This corresponds well to our previous solution, without the use of the
quadratica formula. We have thus found a general method for finding approximate
solutions to polynomials of any degree with a small parameter $\epsilon$.

\section{A singular perturbation}

In the last section we dealt with a so called regular perturbation problem. In this section
we will deal with a singular perturbation problem. What is the difference? In a
singular perturbation problem, the small $\epsilon$ \textit{matters} for the
solution. In general, singular problems are interesting precisely
because their solutions can change a lot with just a small change in
circumstances.

As before, we will use a quadratic equation to illustrate how this works.

\begin{equation}
\epsilon x^2 - 2 x + 1 = 0
\end{equation}

As before, we take $\epsilon$ to be a very small number. This equation has the
following solutions.

\begin{align}
x_1 &= \frac{1 + \sqrt{1 - \epsilon}}{\epsilon}, \\
x_2 &= \frac{1 - \sqrt{1 - \epsilon}}{\epsilon}.
\end{align}

The first thing we notice is that even though $\epsilon$ is very small, we can't
set it to zero. If we were to do it in (14), we would only get a one degree
polynomial. The fact that we are losing solutions is a qualitative change, and
is indicative that we are dealing with a singular perturbation problem.

Let's try using the same method a we did before. We assume the solution can
be expressed in the form of a power series of $\epsilon$. Inserting this in our
equation gives us the following.

\begin{equation}
\epsilon (a_0 + a_1 \epsilon + a_2 \epsilon^2)^2 - 2(a_0 + \epsilon a_1 +
a_2 \epsilon^2 + ...) + 1 = 0
\end{equation}

which gets expanded into.

\begin{equation}
(- 2 a_0 + 1) + (a_0^2 - 2 a_1) \epsilon + (2 a_0 a_1 -2 a_2) \epsilon^2 +
O(\epsilon^3)
\end{equation}

and leads to.

\begin{align}
- 2 a_0 + 1 &= 0, \\
a_0^2 - 2 a_1 &= 0, \\
2 a_0 a_1 -2 a_2 &= 0
\end{align}

$a_0 = \frac{1}{2}$, $a_1=\frac{1}{8}$, $a_2= \frac{1}{64}$, but this gives us
only one of the roots, $x_1$.

\begin{equation}
x_1 = \frac{1}{2} + \frac{1}{8} \epsilon + O(\epsilon^2)
\end{equation}

By the Fundamental Theorem of Algebra, we would expect to see two solutions.
What happened to the other root? We missed it because it's not on the form of
a perturbation series. So what do we do?

The key here is that we can do a change in variable to turn the problem into a
regular perturbation problem.

\begin{equation}
x(\epsilon) = \frac{y(\epsilon)}{\delta(\epsilon)}
\end{equation}

Here we are treating $x$ as a function, $y(\epsilon)$ is $O(1)$ and we want to
determine the re-scaling factor function $\delta(\epsilon)$. Our original
equation becomes

\begin{equation}
\frac{\epsilon}{\delta^2} y^2 - \frac{2}{\delta} y + 1 = 0
\end{equation}

Our goal is to simplify this equation. We do this by dropping relatively
insignifcant terms. As we have seen, it turns out that the first term in the
equation is not insignificant, so we have to leave it in. Is there some other
term that we, to a first approximation, can drop?

TODO: Cite for method of dominant balance.

All the three terms in the above equation have some order of magnitude. The method
of dominant balance tells us to look for pairs that balance, where balance means
they are of the same order of magnitude. We have already determined that the
first term can't be dropped, so we have two options:

\textbf{Case 1.} $\frac{\epsilon}{\delta^2} y^2$ balances 1, with $\frac{2}{\delta}$
being relatively insignificant.

$\frac{\epsilon}{\delta^2} = 1$ implies $\delta = \epsilon^{\frac{1}{2}}$. But
then $\frac{2}{\delta} = \frac{2}{\sqrt{\epsilon}}$ which isn't small as
$\epsilon \to 0$.

\textbf{Case 2.} $\frac{\epsilon}{\delta^2} y^2$ balance $\frac{2}{\delta} y$,
with 1 being relatively insignificant.

This means $\frac{\epsilon}{\delta^2} = \frac{1}{\delta}$ which implies that
$\delta = \epsilon$. This seems correct as both expressions are
$O(\frac{1}{e})$, and 1 is relatively small compared to that when $\epsilon \to
0$. We have

TODO: Fix Latex small paranthesis.

\begin{align}
P(x) = \epsilon x^2 - 2x + 1, \\
\epsilon P(\frac{y}{\epsilon}) = y^2 - 2 y + \epsilon
\end{align}

Where the last part is exactly the same as our regular perturbation problem.
Like before, this gives us

\begin{align}
y_1 &= \frac{1}{2} \epsilon + \frac{1}{8} \epsilon^2 + O(\epsilon^3), \\
y_2 &= 2 - \frac{1}{2} \epsilon - \frac{1}{8} \epsilon^2 + O(\epsilon^3)
\end{align}

which means that

\begin{align}
x_1 &= \frac{1}{2} + \frac{1}{8} \epsilon + O(\epsilon^2), \\
x_2 &= \frac{2}{\epsilon} - \frac{1}{2} + \frac{1}{8} \epsilon + O(\epsilon^2)
\end{align}

The second root is our missing solution, and it corresponds to $x \to \infty$ as
$\epsilon \to 0$. This is the essence of singular perturbation theory - to find
the singular behavior and do a change of variable to turn it into a regular
perturbation problem.

\section{ODE and boundary theory}

TODO: Section on basics of boundary theory, intuition? Which book? ~ Ref Prandtl
handwavey what boundary means.

\subsection{Outer solution}

We are now going to look at a differential equation.

\begin{equation}
\epsilon y'' + 2 y' + y, y(0)=0, y(1)=1, 0 < x < 1
\end{equation}

If we naively set $\epsilon = 0$ we see that the resulting equation $2 y' + y$
has the general solution $C e^{- \frac{1}{2}x}$. This can't satisfy both
boundary conditions at once. If it satisfies $y(0)=0$ we have $y=0$ as the only
solution, and if it satisfies $y(1)=1$ we have

\begin{equation}
y_O=e^{\frac{1}{2} (1 - x)}
\end{equation}

TODO: Why is this assumption legit? Ask Y

We are going to assume this is a good approximation somewhere. We are also going
to assume this is valid for the $y(1)=1$ boundary condition. We call this out
outer solution.

Like the example in the last section, we missed something when we set $\epsilon
= 0$. We have found one of the two solutions and now we want to the find the
other one with the help of pairwise balancing.

\subsection{Inner solution}

We are assuming the boundary layer is near 0, and that it has a thickness
$\delta(\epsilon)$. We introduce a re-scaling variable

\begin{equation}
\overline{x} = \frac{x}{\delta}
\end{equation}

Having two scales like this is typical for singular perturbation problems. Our
original equation (31) becomes

\begin{equation}
\frac{\epsilon}{\delta^2} \frac{d^2y}{d\overline{x}^2} + \frac{2}{\delta}
\frac{dy}{d\overline{x}} + y = 0
\end{equation}

We now do pairwise balancing. We have the following orders of
magnitude

\begin{equation}
\frac{\epsilon}{\delta^2}, \frac{1}{\delta}, 1
\end{equation}

We must have $\frac{\epsilon}{\delta^2}$ present, so the question is if it
balance one of the other terms with one being insignificant, and if so, which one
is insignificant. There are two cases.

TODO: Latex imply, x2.

\textbf{Case 1}
\begin{equation}
\frac{\epsilon}{\delta^2} ~ 1 => \delta = \epsilon^{\frac{1}{2}},
\end{equation}

but then $\frac{1}{\delta}$ is big compared to 1.

\textbf{Case 2}
\begin{equation}
\frac{\epsilon}{\delta^2} ~ \frac{1}{\delta} => \delta = \epsilon,
\end{equation}

in which case 1 is indeed insignificant.

We have, multiplying by $\epsilon$, the inner equation:

\begin{equation}
\frac{\delta^2 y}{\delta \overline{x}^2} + 2 \frac{dy}{d\overline{x}} + \epsilon
y = 0
\end{equation}

To a first approximation we can neglect the last term

\begin{equation}
\frac{\delta^2 y}{\delta \overline{x}^2} + 2 \frac{dy}{d\overline{x}} = 0
\end{equation}

Together with the other initial condition $y(0)=0$ we get the general inner
solution, valid in a region of thickness $\epsilon$ close to $x=0$

\begin{equation}
y_I = C(1 - e^{-2\overline{x}})
\end{equation}

We now have an outer and inner solution, and we turn to matching these to
determine the constant $C$.

\subsection{Matching}

The idea behind matching is that there is some edge or region between the inner
and outer solution. A region where both solutions are valid approximations.

If we imagine a particle tracing the x-axis rightward, as we exit the boundary
layer, i.e. as $\overline{x} \to \infty$, the value of $y_I$ should be equal to
the value of $y_O$ as $x \to 0$, that is

TODO: Wrong Latex format of limit.

\begin{equation}
\lim \overline{x} \to \infty y_I = \\lim x \to 0+ y_O \iff 
\lim \overline{x} \to \infty C(1 - e^{-2\overline{x}}) = \lim x \to 0+ e^{\frac{1}{2}(1-x)}
\end{equation}

This is called a \textit{matching condition}. The right-hand side is
$e^{\frac{1}{2}}$, and thus the left-hand side gives us that $C =
e^{\frac{1}{2}}$.

\subsection{Uniform approximation}

We now have two separate approximations. We would like to have one single
approximation that is valid everywhere. We do this by adding the two
approximations together and removing their common part.

\begin{equation}
y_U = y_o(x) + y_I(\frac{x}{\delta}) - e^{\frac{1}{2}}
\end{equation}

The common part comes from the matching condition in the previous section. In
the inner region the other terms are negliable, and vice versa in the outer
region.

\begin{align}
y_U &= e^{\frac{1}{2}(1-x)} e^{\frac{1}{2}}(1 - e^{\frac{-2x}{\epsilon}}) -
e^{\frac{1}{2}} \\
    &=  e^{\frac{1}{2}}(e^{\frac{-x}{2}} - e^{\frac{-2x}{\epsilon}})
\end{align}

Which is our uniform approximation for the whole solutions.

TODO: Compare vs exact solution and graph them. Mathematica or XPP.

We can compare this to the exact solution and see that it is indeed a good
approximation.

\section{Total Quasi Steady State}

\subsection{A biological problem}

In enzyme kinetics we often come across chemical reactions like

TODO: Fix left right arrows.

\begin{equation}
E + S \leftarrow k_1 \rightarrow k_{-1} C \rightarrow k_2 E + P
\end{equation}

This is an enzyme E and substrate S that combine, reversibly, to form a complex
C, which in turn gives us a product and enzyme back. $k_1, k_{-1}, k_2$ are rate
constants. We won't bother with the biological etails too much, but instead look
at how we can analyze the above as a dynamical system.

The \textit{law of mass action} tells us that two molecules a and b forming
complex c, $a+b \rightarrow k c$ are governed by $\frac{dc}{dt} = kab$. That is,
the more concentration of a or b we have, the faster they turn into complex c.
This has been experimentally verified in 1867 by Guldberg and Waage, and many
times since.

With this law, we can translate our chemical reaction a system of differential
equations.

\begin{align}
\frac{dE}{dt} &= -k_1(E+S) + k_{-1}C + k_2C, \\
\frac{dS}{dt} &= -k1(E+S) + k_{-1}C, \\
\frac{dC}{dt} &= k_1(E+S) - k_{-1}C - k_2C, \\
\frac{dP}{dt} &= k_2C
\end{align}

In the study of enzyme kinetics, we normall have some initial conditions to help
us out: $S(0) = S_0, E(0) = E_0, C(0)=0, P(0)=0$.

We see immediately that

\begin{equation}
\frac{dE}{dt} + \frac{dC}{dt} = 0
\end{equation}

Together with the initial conditions we get that

\begin{equation}
E + C = E_0
\end{equation}

Which we can use to eliminate E from the equation. This is called a conservation
law.  We also don't care about P, as it doesn't feed back into the other
equations. We are left with

\begin{align}
\frac{dS}{dt} &= -k_1(E_0 - C)S + k_{-1}C, \\
\frac{dC}{dt} &= k_1(E_0 - C)S - k_{-1}C - k_2 C
\end{align}

This is a two-dimensional system, which is much easier to deal with. Can we do
better? It turns out we can.

Very often, in practice, the substrate has a much higher concentration than enzyme. This
means that C can be treated as constant after a short initial period. By
exploiting this fact, we can reduce the system to one equation. This is called
the \textit{Quasi-Steady State approximation}.

Assume that C is constant $\iff \frac{dC}{dt} = 0$ after a short period of time,
then we can could write, as $\frac{dC}{dt}$ can be treated as 0,

\begin{equation}
\frac{dS}{dt} = -k2 C
\end{equation}

with

\begin{equation}
C = \frac{E_0 S}{K_m +S}
\end{equation}

where $K_m = \frac{k_{-1} + k_2}{k_1}$ is the so called {Michaelis-Menten
constant}.

\begin{equation}
\frac{dS}{dt} = \frac{-k_2 E_0 S}{K_M + S}
\end{equation}

together with $S(0) = S_0$ as initial condition is what is normally considered
the QSSA, valid after some short amount of time has passed.

We can justify this with a procedure similar to the previous section, as done by
for example Lin and Segel \cite{lin1974mathematics}. Instead we are going to
look at a slight variation of this, namely what happens when the amount of
substrate isn't much bigger than the amount of enzymes. This is the
\textit{Total Quasi Steady State assumption}, discovered by Borghans and Segel
1996 \cite{borghans1996extending}.

\subsection{TQSSA}

The basic idea behind TQSSA is to perform a change of variable

\begin{equation}
\overline{S} = S + C
\end{equation}

This change of variable makes the approximation valid for more parameters. We
get the following system of equations from (52), (53) and (57)

\begin{align}
\frac{d\overline{S}}{dt} &= - k_2 C, \\
\frac{dC}{dt} &= k_1(E_0-C)(\overline{S}-C)-(k_{-1}+ k_2) C
\end{align}

With initial conditions $\overline{S}(0)=S_0, C(0)=C$, these are the rate
equations for the TQSSA.

TODO: Show and explain phase plane for intuition. Mathematica or XPP.

TODO: Justify time scales chosen? Ask Y.

Borghans et al. found these two following time scales,

\begin{align}
t_c &= \frac{1}{k_1(E_0+S_0+K_M)}, \\
t_{\overline{s}} &= \frac{E_0+S_0+K_m}{k_2+E_0}
\end{align}

$K_M = \frac{k_{-1}+k_2}{k_1}$ as before. We will now analyze this like we did
in the last section. But first, we have to scale the system.

\subsection{Scaling}

We will now introduce scaled, dimensionless variables. We have the two times:

TODO: Include Segel def. of scaling (p11) (Slemrod ~441p)

\begin{align}
t_f &= \frac{t}{t_c}, \\
t_s &= \frac{t}{t_{\overline{S}}}
\end{align}

$t_f$ is the fast time scale, and $t_s$ the slow one. We also scale C and
$\overline{S}$ by their maximum:

\begin{align}
c &= \frac{C}{C_0}, \\
s &= \frac{\overline{S}}{S_0}
\end{align}

$S_0$ is max since it starts from some constant and then turns into complex. $C_0$
is max derived by Borghans et al. \cite{borghans1996extending} as

TODO: Show how Borghans derive this? Ask Y.

\begin{equation}
C_0 = \frac{E_0 S_0}{E_0 + S_0 + K_m}
\end{equation}

\subsection{Outer solution}

We now have a mathematical formulation of the problem and we just need a small
parameter $\epsilon$ to begin looking for the outer solution, where the complex
changes slowly.

A necessary condition for TQSSA to hold is $0 < \epsilon \leq 1$

TODO: Why don't Khoo use $\ll$? Odd. Ask Y.

\begin{equation}
\epsilon = \frac{t_f}{t_s}
\end{equation}

This comes from Borghans. \textit{Assuming} $t_f \ll t_s$ we then get, for the
outer region.

Using c and $t_s$ we get, using the procedure used by Khoo and Heglund
\cite{khoo2008total},

\begin{equation}
\frac{dC}{dt_s} = \frac{t_{\overline{S}}}{c_0} k_1 [E_0 S_0 s - (E_0 + S_0 s +
K_M) C_0 c + (c c_0)^2]
\end{equation}

Recall that

\begin{equation}
\epsilon = \frac{t_f}{t_s} = \frac{k_2 E_0}{k_1(E_0+S_0+K_m)^2}
\end{equation}

so we can rewrite the above as

\begin{equation}
... = \frac{1}{\epsilon} (s - \frac{E_0+S_0 s + k_m}{E_0 + S_0+ K_M}C + \gamma^2
c)
\end{equation}

\begin{equation}
\gamma = \frac{E_0 S_0}{(S_0 + K_m + E_0)^2}
\end{equation}

TODO: Check algebra and lower case c, s.

TODO: This reasoning is wrong, It's due to Tikhonov theorem (C433)

As $\epsilon \to 0, \frac{1}{\epsilon} \to \infty$, but C(0) = 0 so

\begin{equation}
s - \frac{E_0 + S_0 s + K_m}{E_0 + S_0 + K_M}C + \gamma c^2 = 0
\end{equation}

TODO: What is it? How do they work? Take for granted or explain?

By using Pade approximants we can get an approximation for the outer
solution

\begin{equation}
c_0  = \frac{E_0 + S_0 + K_m}{E_0 + S_0 s + K_m} s
\end{equation}

This is the outer solution. If we substitute this into s we get an
expression for outer solution of s.

TODO: Check algebra. Mathematica?

\begin{equation}
\frac{ds}{dt_s} = \frac{-E_0 + S_0 + K_m}{E_0 + S_0s + k_m}s, s(0) = 1
\end{equation}

Solving this, we get the outer solution for s, according to Khoo:

TODO: Derive yourself? Same below

\begin{equation}
(E_0 + K_m) ln s_O(t_s) + s_o(s_o(t_s) - 1) + (E_0 + S_0 + K_m) t_s = 0
\end{equation}

\subsection{Inner solution}

TODO: Show intermediate steps. Which ones?

Using the outer time scale to satify the other initial conditions,

\begin{equation}
\frac{dS}{dt_f} = \frac{-k_2 E_0}{k_1(E_0 + S_0 + K_M)^2} c, s(0) = 1
\end{equation}

As $\epsilon \to 0$, the above becomes $\frac{dS}{dt_f} = 0$, i.e constant.

\begin{equation}
s_I = s(0) = 1
\end{equation}

For c:

\begin{equation}
\frac{dc}{dt_f} = 1 - c - \gamma c^2, c(0)=0
\end{equation}

TODO: How solve this? Mathematica?

Khoo et al. finds that the solution is

\begin{equation}
c_I = \frac{2[exp(\sqrt{4\gamma-1}t_f)-1]}{(1-\sqrt{1-4\gamma}
exp(\sqrt{4\gamma-1}t_f) - (1+\sqrt{1-4\gamma}}
\end{equation}

where c(0)=0 holds.

TODO: Check this.

\subsection{Matching}

As before, we are looking for a common limit, and our matching condition as
$\epsilon \to 0, t_f \to \infty, t_s \to 0$ is

\begin{equation}
\lim \epsilon 0 [y_o(t_s)|t_s=0] = \lim \epsilon \to 0[y_I(t_f)|t_f=\infty]
\end{equation}

For s:

\begin{equation}
\lim \epsilon 0 [y_o(t_s)|t_s=0] = \lim \epsilon \to 0[y_I(t_f)|t_f=\infty] = 1
\end{equation}

For c:

TODO: Perform this limit check, help from Mathematica? Can I find by hand? Khoo
finds

\begin{equation}
... = \approx 2(1+\sqrt{1-4\gamma})^{-1}
\end{equation}

\subsection{Uniform approximation}

Like before, we take the common part and subtract the difference.

\begin{align}
s_u &= s_O + s_I - 1 = s_O, \\
c_u &= c_O + c_I - 2(1 + \sqrt{1-4\gamma})^{-1}
\end{align}

TODO: Phase plane / numerical? Same as before. But compared to what?

\bibliography{references}

\end{document}
